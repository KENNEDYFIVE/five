\documentclass{article}
\begin{document}
\begin{center}
\begin{tabular}{|l|l|l|c|}
\hline NAME  & REG NO & STD NO \\\hline
MULINDWA MATOVU KENNEDY& 16/U/19713/PS & 216021701 \\\hline
\end{tabular}
Lecturer: ERNEST MWEBAZE \\
\end{center}
\newpage
\begin{center}
\textbf{LITERATURE REVIEW FOR ANDROID}\\
\end{center}
\paragraph
 Technology changes so rapidly that it’s hard to keep up with it sometimes. The use of
smartphones as a research tool is very promising because it has the chance to adapt to the rapid advancement of technology. This means that this topic of how smartphones improve research can help professors, students, researchers, and even just anyone who owns a smartphone. Research tools used to rely on desktop computers and specialized operating systems for every field. With the introduction of the android operating system along with the ios operating system, applications are widely available and easily distributed to thousands of people at an instant. The
smartphone’s small form factor makes it extremely portable, allowing people to carry it on their
 person at all times. Most of the recent smartphones are more powerful than some laptops and even computers of today. This makes them a very valuable tool that can be adapted to the research environment in any field of study. Research has always been a very important aspect to universities and life in general. Research is the means by which our society progresses. There has always been a wide variety of tools that are used for research and even those tools are replaced eventually by cheaper and more  precise instruments. Cell phone technology is constantly changing and evolving, sometimes so
fast it’s hard to keep up with the advancements that are made every day (Chappel, Castro 
Palacio, Di Noia, Gold, Kim, Nacci, Stopczynski). The smartphone is a perfect example of this; the first smartphone was introduced in 2007 by Apple and now in 2014 over 22 percent of people in the world own a smartphone (Chappel, Kim, Nacci, Sans, Suarez-Tangil, Stopczynski, Young, Young-Seol). In the United States alone, 61 percent of the population uses a smartphone on a daily  basis (Gold, 2012). With so many people in the world using a smartphone, it only makes sense to make it into one of the most versatile tools in the market. Smartphones are packed with many different sensors that cover multiple types of data such as accelerometers and light-sensors to GPS and gyroscopes. The operating systems that run on them are also very easy for almost anyone to use. When the multitude of sensors and data-recording devices inside of these smartphones mix with the easy to use operating system, you get a device that seems like it was created for research. The truth of the matter though is that these devices are only just beginning to be used for research purposes. The earliest study I could find about a smartphone being used as a research tool was from 2012 (Young, 2012). But in every article I read the smartphone has been shown to be vital to research (Castro-Palacio, Chappel, Di Noia, Kim, Montoya, Sans, Stopczynski, Young, Young-Seol). The smartphone has the chance to revolutionize how people view research by being able to allow anyone to perform it. With only the download of an app people can do their own
research on any subject they want. And with over a fifth of the world’s populatio
n having access to a smartphone, the amount of scientific breakthroughs that could occur through self-guided research is staggering.
\newpage
\section{CITATION}
\paragraph{•}
Castro-Palacio, Juan Carlos; Velázquez-Abad, Luisberis. "Using A Mobile Phone Acceleration Sensor In Physics Experiments On Free And Damped Harmonic Oscillations."
 American  Journal Of Physics
 81.6 (2013): 472-475. Print. Chappel, Roger. "An educational platform to demonstrate speech processing techniques on Android based smart phones and tablets."
Speech Communication
. 57. (2014). Di Noia, Tommaso. "An end stage kidney disease predictor based on an artificial neural networks ensemble."
 Expert Systems with Applications
. 40.11 (2013). Gold, Steve. "Hacking On The Hoof."
Engineering \& Technology (17509637)
 7.3 (2012): 80-83. Kim, Juseuk. "Adapting Smartphones as Learning Technology in a Korean University."
 Journal of Integrated Design and Process Science
. 17.1 (2013). Montoya, Francisco. "A monitoring system for intensive agriculture based on mesh networks and the android system."
Computers \& Electronics in Agriculture
. 99. (2013): 14-20.
\end{document}
